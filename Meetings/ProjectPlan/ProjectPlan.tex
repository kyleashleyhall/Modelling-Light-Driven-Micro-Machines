%%
%% This is file `./samples/minutes.tex',
%% generated with the docstrip utility.
%%
%% The original source files were:
%%
%% meetingmins.dtx  (with options: `minutes')
%% ----------------------------------------------------------------------
%% 
%% meetingmins - A LaTeX class for formatting minutes of meetings
%% 
%% Copyright (C) 2011-2013 by Brian D. Beitzel <brian@beitzel.com>
%% 
%% This work may be distributed and/or modified under the
%% conditions of the LaTeX Project Public License (LPPL), either
%% version 1.3c of this license or (at your option) any later
%% version.  The latest version of this license is in the file:
%% 
%% http://www.latex-project.org/lppl.txt
%% 
%% Users may freely modify these files without permission, as long as the
%% copyright line and this statement are maintained intact.
%% 
%% ----------------------------------------------------------------------
%% 
\documentclass[11pt]{meetingmins}

\usepackage[english]{babel}
\usepackage[utf8x]{inputenc}
\usepackage[T1]{fontenc}
\usepackage{etoolbox}
\usepackage{subcaption}
\setlength{\oddsidemargin}{0pt} % give 2.54cm (1 inch) left margin
\setlength{\headheight}{15pt} 
\usepackage{parskip}
\setlength{\parindent}{0pt}
\usepackage{amsmath}
\usepackage{amssymb}
\usepackage{amsfonts}
\usepackage{graphicx}
\usepackage[colorinlistoftodos]{todonotes}
\usepackage[colorlinks=true, allcolors=black]{hyperref}
\usepackage{pdfpages}
\usepackage[titletoc]{appendix}
\usepackage{tocbibind}
\usepackage{tocloft}
\usepackage{xpatch}
\usepackage{float}

%% Optionally, the following text could be set in the file
%% department.min in this folder, then add the option 'department'
%% in the \documentclass line of this .tex file:
%%\setcommittee{Department of Instruction}
%%
%%\setmembers{
%%  \chair{B.~Smart},
%%  B.~Brave,
%%  D.~Claire,
%%  B.~Gone
%%}

\setcommittee{Modelling Light-Driven Micro-Machines - Project Plan}

\setmembers{
  Marius van Laar,
  Kyle Hall,
}

\setdate{November 29, 2017}

\setpresent{
  Marius van Laar,
  Kyle Hall,
}

\begin{document}
\maketitle

\section{1. Investigate the effects of optics on a fixed particle system}

\begin{itemize}

\item Using DDA calculate the forces and torques on the macroscopic particle

\begin{itemize}

\item Use ADDA

\end{itemize}

\item Apply T-Matrix theory for rotationally symmetric particles

\item Use both Gaussian and Planar beams (Planar beams will allow us to clarify our force calculations with those calculated within ADDA)

\item Evaluate the optimum calculation of forces and torques, removing truncation errors

\item Vary the propagation direction of the beam for a non realistic system

\item Use the following objects in this sequence, to illustrate evolution of understanding:

\begin{enumerate}

\item Sphere

\item Rotationally symmetric object (simple)

\item Rotationally symmetric object (complex)

\item Non Rotationally symmetric object

\end{enumerate}

\end{itemize}

\section{2. Investigate external effects on particles}

\begin{itemize}

\item Taking the simple spherical particle and seeing how it behaves as a under different conditions (i.e. Friction, Gravity, etc.)

\item Vary environmental conditions (i.e. Temperature, Refractive index of the medium, etc.)

\item Vary the propagation direction of the beam for a real system

\item Consider how to translate between the different reference frames of the:

\begin{itemize}

\item Particle

\item Beam

\item Laboratory

\end{itemize}

\item Consider more complex objects within the same scenarios illustrated above

\end{itemize}

\section{3. Investigate realistic systems}

\begin{itemize}

\item Replicate an experiment currently being undertaken by postgraduate students, to determine the extent to which microscopic arrow heads are moving in a system. Compare and contrast our results with the physical results attained in the laboratory

\item Consider an optical tweezer arrangement to manipulate realistic particles, compare to previous experiments

\item Create a propeller system that replicates physical motion to an applied particle

\item Create a drill simulation, whereby we move a formed drill bit to create holes in a material using a beam or optical beams arrangement

\end{itemize}

\end{document}
%% 
%% Copyright (C) 2011-2013 by Brian D. Beitzel <brian@beitzel.com>
%% 
%% This work may be distributed and/or modified under the
%% conditions of the LaTeX Project Public License (LPPL), either
%% version 1.3c of this license or (at your option) any later
%% version.  The latest version of this license is in the file:
%% 
%% http://www.latex-project.org/lppl.txt
%% 
%% Users may freely modify these files without permission, as long as the
%% copyright line and this statement are maintained intact.
%% 
%% This work is "maintained" (as per LPPL maintenance status) by
%% Brian D. Beitzel.
%% 
%% This work consists of the file  meetingmins.dtx
%% and the derived files           meetingmins.cls,
%%                                 sampleminutes.tex,
%%                                 department.min,
%%                                 README.txt, and
%%                                 meetingmins.pdf.
%% 
%%
%% End of file `./samples/minutes.tex'.
